\documentclass[12pt, titlepage]{article}

\usepackage{booktabs}
\usepackage{tabularx}
\usepackage{hyperref}
\hypersetup{
    colorlinks,
    citecolor=black,
    filecolor=black,
    linkcolor=red,
    urlcolor=blue
}
\usepackage[round]{natbib}

\title{SE 3XA3: Software Requirements\\Spann}

\author{Team 5
		\\ Christopher Stokes | stokescd
		\\ Varun Hooda | hoodav
}

\date{\today}

%% Comments

\usepackage{color}

\newif\ifcomments\commentstrue

\ifcomments
\newcommand{\authornote}[3]{\textcolor{#1}{[#3 ---#2]}}
\newcommand{\todo}[1]{\textcolor{red}{[TODO: #1]}}
\else
\newcommand{\authornote}[3]{}
\newcommand{\todo}[1]{}
\fi

\newcommand{\wss}[1]{\authornote{blue}{SS}{#1}}
\newcommand{\ds}[1]{\authornote{red}{DS}{#1}}
\newcommand{\mj}[1]{\authornote{red}{MSN}{#1}}
\newcommand{\cm}[1]{\authornote{red}{CM}{#1}}
\newcommand{\mh}[1]{\authornote{red}{MH}{#1}}

% team members should be added for each team, like the following
% all comments left by the TAs or the instructor should be addressed
% by a corresponding comment from the Team

\newcommand{\tm}[1]{\authornote{magenta}{Team}{#1}}


\begin{document}

\maketitle

\pagenumbering{roman}
\tableofcontents
\listoftables
\listoffigures

\begin{table}[bp]
\caption{\bf Revision History}
\begin{tabularx}{\textwidth}{p{3cm}p{2cm}X}
\toprule {\bf Date} & {\bf Version} & {\bf Notes}\\
\midrule
Oct. 1, 2016 & 1.0 & Initial Changes\\
Oct. 11, 2016 & 1.1 & Finishing Changes\\
\bottomrule
\end{tabularx}
\end{table}

\newpage

\pagenumbering{arabic}

\section{Project Drivers}

  \subsection{The Purpose of the Project}
  The purpose of the project is to develop a web browser based Python IDE
  application. The application will provide an environment similar to desktop
  based integrated development environments, but with the convenience of a
	seamless experience regardless of their operating system and hardware
	platform.

  \subsection{The Stakeholders}

    \subsubsection{The Client}
    The client for whom this application is being developed is Dr. Smith, the
    professor of Software Engineering 3XA3.

    \subsubsection{The Customers}
    The customers of the application will be python developers looking for a
    convenient platform that allows them to develop from almost anywhere, on
    almost anything, as long as they have an internet connection.

    \subsubsection{Other Stakeholders}
		Other stakeholders include students and teachers looking for a platform to
		learn and teach fundamentals of programming without the trouble of setting
		up a programming environment by themselves.

  \subsection{Mandated Constraints}
  The project needs to run on software and hardware that McMaster University
  has accesses to and has the licenses for.

  \subsection{Naming Conventions and Terminology}
	The following terms may be used throughout this document:

	\begin{description}
		\item [Us/We] The members of the group who are developing this project.
		\item [Libre] Free and Open Source Software. Specifically, free in terms of
			freedom as appose to price; thus freedom to modify, view and possibly
			redistribute the software.
		\item [Application] The Spann Online IDE application.
		\item [IDE] Integrated Development Environment is a application for
			developing software within a standalone suite of tools.
		\item [SaaS (Software as a Service)] A model of computation in which a piece
			of software if provided to the user as a service rather than a product.
		\item [Virtualization] A method of computation in which a software utility
			isolates a part of the computational space from the rest of the system to
			increase modularity and security of the system as a whole.
	\end{description}

  \subsection{Relevant Facts and Assumptions}
  The application will assume the user has a modern, HTML5 compatible browser
  that has JavaScript enabled. The application will be tested to ensure it is
	functional on the major modern browsers (see appendix).

\section{Functional Requirements}

  \subsection{The Scope of the Work and the Product}

    \subsubsection{The Context of the Work}
		The work will be mainly software design, implementation using the chosen
		programming languages and testing to ensure the application works as
		specified.

    \subsubsection{Work Partitioning}
		In general the split of work is approximately equal in client and server
		side code. The difference in who is responsible for the underlining
		architecture development and who is responsible for the application
		specific development. In general the underlining setup and architecture
		development is handled by Christopher Stokes and the application specific
		development by Varun Hooda.

    \textbf{Christopher Stokes:}
    \begin{itemize}
        \item Project and code design
        \item Database design
        \item SQL code generation framework
        \item UI framework design
        \item UI design
        \item Testing
    \end{itemize}


    \textbf{Varun Hooda:}
    \begin{itemize}
        \item Database design
        \item UI design and development
        \item API development
        \item Client and server side algorithm design and development
        \item Testing
    \end{itemize}

    \subsubsection{Individual Product Use Cases}
		\textbf{Algorithm Testing:} A main use case of this project is to be able
		to design individual parts of larger projects or algorithms without it
		interacting with the larger project. Because of how quick this project is a
		creating  projects it makes it incredibly easy to test small sections of
		code during development. This can be done in a project or in the console.
		\newline\newline
		\textbf{Full Project Development:} This project can be used to develop full
		projects with virtual know limitations on what can be supported.  There are
		a number of reason this would be done, such as it means the developer does
		not need to create a local environment and is able to develop on any
		deceive with access to the web.

  \subsection{Functional Requirements}
	\begin{description}
		\item [Mode] The application shall have two modes of operation U and O (see
			appendix).
		\item [Editing] The application shall allow the user to edit and
			save new and existing files (U mode).
		\item [Editing Support] The application shall provide the user with
			language specific suggestions while the user is editing a file (U or O).
		\item [Refactoring] The application shall allow the user to refactor parts
			of the file(s) they are working on (U or O).
		\item [File Handling] The application shall manage (view, store, track)
			files created by the user (U).
		\item [Code execution] The application shall allow the user to execute a
			file containing python code and the application will display the output
			of the execution to the user (U or O).
		\item [Shell Interpreter] The application shall provide the user with a
			interactive shell for executing individual commands (specifically python)
			and present the user with the output of the command (U or O).
		\item [Networking] The application shall execute the code and store user
			created data (file, metadata, user account data) on the server and
			forward the output and other data to the browser via a network
			connection.
		\item [Accounts] The application shall allow new users to create an account
			and allow existing users to login into their user account (O).
		\item [Account Mangement] The application shall allow the user to manage
			their accounts through a user interface on the application (manage
			account details such as password, email, etc.) (U).
	\end{description}

\section{Non-functional Requirements}

  \subsection{Look and Feel Requirements}
	This application needs to maintain a unified enterprise look and feel. A
	major goal of this application is to increase development speed, this means
	it is incredibly important that the user experience is clean. To achieve
	this, options such as save, properties, and right-click must be constant
	throughout the application. As a target, this application must respond and
	feel like a desktop application while running in browser meaning all actions
	that one would expect to work in a desktop IDE such as Eclipse must work in
	browser.

  \subsection{Usability and Humanity Requirements}
		Users of this project are generically of a technical background and the
		nature of the work  is technical. This means the product is able to be
		designed in a more technical nature but it is still import that the product
		is usable by ensure a consistent design and button location across all
		screens.

  \subsection{Performance Requirements}
    \subsubsection{Client}
		The client side UI must be quick and responsive as well as not using a lot
		of resources as it runs in the browser. It is most important that the
		application can run asynchronously at all times to not lock up the browser.

    \subsubsection{Server}
		For the server it is very important that it is fast and is able to handle a
		lot of requests. It also needs to run its processes in a proper
		asynchronous manner as the processes are long running.

  \subsection{Operational and Environmental Requirements}
	The environmental impact of an engineering project should always go under
	considering when designing and implementing a project. For this project we
	aim to reduce the environmental impact of this project by as much as
	possible. This means the application will have to meet a number of
	operational requirements,
  \begin{description}
		\item [Power Usage] The application needs to be energy efficient during
			operation. This application needs to reduce power usage wherever possible
			using optimization techniques and efficient algorithms.
		\item [Hardware] The application needs to be able to perform on energy
			efficient (possibly low power) hardware.
  \end{description}

  \subsection{Maintainability and Support Requirements}
	It is highly important that the source for this project is maintainable as
	the features of the IDE will grow as more advanced tools are added to
	encompass or use cases and parts of the development processes. Furthermore
	web APIs and support changes incredibly fast so it is important that the code
	stays up to date to support the best technologies.

  \subsection{Security Requirements}
  This application should provide a secure platform and carry out its
  functionality in a secure manner. This means the application needs to meet
  the following a set of security requirements:
  \begin{itemize}
    \item The application cannot allow anyone except the owner of the account
      to view/modify to the files associated with the account and the settings
      associated with the account.
    \item The application does not allow anyone to intercept the data while it
      is being transferred over the network.
    \item The application executes user code in secure manner, isolated from
      the rest of the system to ensure any malicious code doesn't compromise
      the security of the system and application.
  \end{itemize}

  \subsection{Cultural Requirements}
  The project may need to be translated or support translation if a significant
  number of users' primary language is not English.

  \subsection{Legal Requirements}
	The project needs to abide by all international and domestic laws, as well as
	all of McMaster University's Policies, Procedures and Guidelines. Depending
	on the license under which this project is released we may also need to
	consider other legal obligations (this is will made clear once a license has
	been chosen).

  \subsection{Health and Safety Requirements}
	The project will attempt to be as ergonomic and health conscious as possible.
	Thus the following will be incorporated into the project:
	\begin{description}
		\item [Night Mode] Allow the user to adjust the applications color scheme
			so it is less stressful on the eyes and easier to view during different
			lighting conditions.
		\item [Minimal User Interface] Allow the user to do as much as possible
			using as little work (clicking, navigating the UI, scrolling) as
			possible.
	\end{description}

	This section is not in the original Volere template, but health and safety
	are issues that should be considered for every engineering project.

\section{Project Issues}

  \subsection{Open Issues}
	The primary issue we mean to address with the product is the lack of online
	fully functional IDE application for the Python programming language. This
	application should be accessible through a modern web browser and be
	presented to the user through the SaaS (Software as a Service) model.

  \subsection{Off-the-Shelf Solutions}
  The are various off-the-shelf solutions available for writing and running
  code on the web browser, but these solutions have little to no support for
  the python programming language besides being able to provide simple code
  completion and ability to execute the code. Thus, for this project, we're
	aiming to both support execution of python code, code completion and add
	IDE-like, developer friendly features (refactoring support and file
	management).

  \subsection{New Problems}
	The follow new problems have arose during the design and development of this
	project,
	\begin{itemize}
		\item Handling python code execution in a secure manner to avoid security
			implication that are present in execution of unknown code.
		\item Maintaining compatibility between multiple browsers and their
			individual web technology implementations.
	\end{itemize}

  \subsection{Tasks}
	The tasks for this project are as follows,
	\begin{itemize}
		\item Design the server and the front end
		\item Design the database schema
		\item Begin implementing the front end
		\item Begin implementing the database
		\item Begin implementing the server
		\item Integrate the front end and server (networking)
		\item Test the system, ensure functionality and security
	\end{itemize}

  \subsection{Migration to the New Product}
	The process of migrating to the new product will not require the user to go
	through any complicated procedure. The process will be signing up for the new
	application, copying the code from the old product or writing new code on the
	new application.

  \subsection{Risks}
  One inherent risk with allowing users to execute code on your servers is the
  user's ability to perform malicious actions. This can result in damage to the
  hardware, the software stack and to the data on the server.  Another risk is
  the possibility of some fault in the system causing user's to lose data or
  the project to lose business critical data or damage the hardware or software
  stack.

  \subsection{Costs}
  The project will be mainly using libre software that
  is available without costs, as well as non-libre software this is available
  to us without cost. If the platform is to be scaled for public usage, the
  project will need to be hosted on some server (or multiple server depending
  on rate of user adoption) which would have an monetary cost.

  \subsection{User Documentation and Training}
	The project will be fully documented, including design documents, testing
	documents, well commented code. The documentation will also include resources
	for new users -- tutorials, guides and user manual. This level of
	documentation should hopefully provide users (as well as developers) enough
	material to user (and perhaps contribute to) the application.

  \subsection{Waiting Room}
	The list of potential features we hope to implement in the future is,
	\begin{itemize}
		\item Moving the application platform from dedicated hardware to a online
			dynamic hosting provider to allow the application to scale to a large
			number of concurrent users.
		\item Support for other languages besides python
		\item Support for large, multi-team, distributed projects
	\end{itemize}

  \subsection{Ideas for Solutions}
	The project members will continue to develop new solutions to the issues
	presented in this section. Some of the potential solutions are,
	\begin{description}
		\item [Secure code execution] Using a virtualization software, the code can
			be executed in a secure manner in an isolated environment on the same
			hardware as the application server.
		\item [Reducing power usage] Using a online dynamic hosting service (such
			as Amazon Web Services) we can ensure our application does not require
			dedicated, fixed hardware. This method allows the application to
			dynamically allocate computational resources based on current usage
			levels, thus reducing redundant, unused computational power and reducing
			power consumption.
	\end{description}

\bibliographystyle{plainnat}

\bibliography{SRS}

\newpage

\section{Appendix}

\subsection{Supported Web Browsers}
\begin{itemize}
	\item Mozilla Firefox
	\item Google Chrome
	\item Microsoft Edge
\end{itemize}

\subsection{Symbolic Parameters}
The application shall have two Modes of operation:
\begin{description}
	\item [(U)ser mode] The user has signed into an account.
	\item [(O)pen mode] The user has not signed in to an account.
\end{description}

\end{document}
