\documentclass[12pt, titlepage]{article}

\usepackage{booktabs}
\usepackage{tabularx}
\usepackage{hyperref}
\hypersetup{
    colorlinks,
    citecolor=black,
    filecolor=black,
    linkcolor=red,
    urlcolor=blue
}
\usepackage[round]{natbib}

\title{SE 3XA3: Software Requirements\\Spann}

\author{Team 5
		\\ Christopher Stokes | stokescd
		\\ Varun Hooda | hoodav
}

\date{\today}

%% Comments

\usepackage{color}

\newif\ifcomments\commentstrue

\ifcomments
\newcommand{\authornote}[3]{\textcolor{#1}{[#3 ---#2]}}
\newcommand{\todo}[1]{\textcolor{red}{[TODO: #1]}}
\else
\newcommand{\authornote}[3]{}
\newcommand{\todo}[1]{}
\fi

\newcommand{\wss}[1]{\authornote{blue}{SS}{#1}}
\newcommand{\ds}[1]{\authornote{red}{DS}{#1}}
\newcommand{\mj}[1]{\authornote{red}{MSN}{#1}}
\newcommand{\cm}[1]{\authornote{red}{CM}{#1}}
\newcommand{\mh}[1]{\authornote{red}{MH}{#1}}

% team members should be added for each team, like the following
% all comments left by the TAs or the instructor should be addressed
% by a corresponding comment from the Team

\newcommand{\tm}[1]{\authornote{magenta}{Team}{#1}}


\begin{document}

\maketitle

\pagenumbering{roman}
\tableofcontents
\listoftables
\listoffigures

\begin{table}[bp]
\caption{\bf Revision History}
\begin{tabularx}{\textwidth}{p{3cm}p{2cm}X}
\toprule {\bf Date} & {\bf Version} & {\bf Notes}\\
\midrule
Oct. 1, 2016 & 1.0 & Initial Changes\\
\bottomrule
\end{tabularx}
\end{table}

\newpage

\pagenumbering{arabic}

This document describes the requirements for Spann.  The template for the Software
Requirements Specification (SRS) is a subset of the Volere
template~\citep{RobertsonAndRobertson2012}.  If you make further modifications
to the template, you should explicity state what modifications were made.

\section{Project Drivers}

  \subsection{The Purpose of the Project}
  The purpose of the project is to develop a web browser based Python IDE
  application. The application will provide an environment similar to desktop
  based integrated development environments but with the convenience of a
  seamless experience regardless of their operating system or hardware platform
  with the only requirements for the user is a modern web browser.

  \subsection{The Stakeholders}

    \subsubsection{The Client}
    The client for whom this application is being developed is Dr. Smith, the
    professor of Software Engineering 3XA3.

    \subsubsection{The Customers}
    The customers of the application will be python developers looking for a
    continent platform that allows them to develop from almost anywhere, on
    almost anything with a internet connection.

    \subsubsection{Other Stakeholders}

  \subsection{Mandated Constraints}
  The project needs to run on software and hardware that McMaster University
  has and has the licenses for.

  \subsection{Naming Conventions and Terminology}

  \subsection{Relevant Facts and Assumptions}
  The application will assume the user has a modern, HTML5 compatible browser
  that has JavaScript enabled. The application will be tested to ensure it is
  functional on the major modern browsers(insert reference to list of browsers
  in appendix here).

  User characteristics should go under assumptions.

\section{Functional Requirements}

  \subsection{The Scope of the Work and the Product}

    \subsubsection{The Context of the Work}

    \subsubsection{Work Partitioning}

    \subsubsection{Individual Product Use Cases}

  \subsection{Functional Requirements}

\section{Non-functional Requirements}

  \subsection{Look and Feel Requirements}
  This application needs to maintain a unified enterprise look and feel. A
   major goal of this application is to increase development speed, this 
   means it is incredibly important that the user experience is clean. To 
   achieve this, options such as save, properties, and right-click must be 
   constant throughout the application. As a target, this application must 
   respond and feel like a desktop application while running in browser 
   meaning all actions that one would expect to work in a desktop IDE such
    as Eclipse must work in browser.

  \subsection{Usability and Humanity Requirements}

  \subsection{Performance Requirements}
    \subsubsection{Client}
    The client side UI must be quick and responsive as well as not using a 
    lot of resources as it runs in the browser. It is most important that the 
    application can run asynchronously at all times to not lock up the browser.
    
    \subsubsection{Server}
    For the server it is very important that it is fast and is able to handle 
    a lot of requests. It also needs to run its processes in a proper 
    asynchronous manner as the processes are long running.

  \subsection{Operational and Environmental Requirements}

  \subsection{Maintainability and Support Requirements}
  It is highly important that the source for this project is maintainable as 
  the features of the IDE will grow as more advanced tools are added to 
  encompass or use cases and parts of the development processes. Furthermore 
  web APIs and support changes incredibly fast so it is important that the code
   stays up to date to support the best technology.

  \subsection{Security Requirements}
  This application should provide a secure platform and carry out its
  functionality in a secure manner. This means the application needs to meet
  the following a set of security requirements:
  \begin{itemize}
    \item The application cannot allow anyone except the owner of the account
      to view/modify to the files associated with the account and the settings
      associated with the account.
    \item The application does not allow anyone to intercept the data while it
      is being transferred over the network.
    \item The application executes user code in secure manner, isolated from
      the rest of the system to ensure any malicious code doesn't compromise
      the security of the system and application.
  \end{itemize}

  \subsection{Cultural Requirements}
<<<<<<< Updated upstream
  {\huge ???????}
=======
  \Huge ???????
>>>>>>> Stashed changes
  The project may need to be translated or support translation if a significant
  number of users' primary language is not English.

  \subsection{Legal Requirements}

  \subsection{Health and Safety Requirements}
<<<<<<< Updated upstream
  {\huge ????????}
=======
  \Huge ????????
>>>>>>> Stashed changes

  This section is not in the original Volere template, but health and safety are
  issues that should be considered for every engineering project.

\section{Project Issues}

  \subsection{Open Issues}

  \subsection{Off-the-Shelf Solutions}
  The are various off-the-shelf solutions available for writing and running
  code on the web browser, but these solutions have little to no support for
  the python programming language besides being able to provide simple code
  completion and ability to execute the code. Thus, for this project, we're
  aiming to both support execution of python, code completion and add IDE-like,
  developer friendly features ({\Huge fill in some IDE-like features we'll
  have})

  \subsection{New Problems}

  \subsection{Tasks}

  \subsection{Migration to the New Product}

  \subsection{Risks}
  One inherent risk with allowing users to execute code on your servers is the
  user's ability to perform malicious actions. This can result in damage to the
  hardware, the software stack and to the data on the server.  Another risk is
  the possibility of some fault in the system causing user's to lose data or
  the project to lose business critical data or damage the hardware or software
  stack.

  \subsection{Costs}
  The project will be mainly using free and open source (libre) software that
  is available without costs, as well as non-libre software this is available
  to us without cost. If the platform is to be scaled for public usage, the
  project will need to be hosted on some server (or multiple server depending
  on user adoption) which would have a regular cost.

  \subsection{User Documentation and Training}
  The project will be fully documented, including design documents, testing documents, well commented code..................

  \subsection{Waiting Room}

  \subsection{Ideas for Solutions}

\bibliographystyle{plainnat}

\bibliography{SRS}

\newpage

\section{Appendix}

This section has been added to the Volere template.  This is where you can
place additional information.

\subsection{Symbolic Parameters}

The definition of the requirements will likely call for SYMBOLIC\_CONSTANTS.
Their values are defined in this section for easy maintenance.


\end{document}
