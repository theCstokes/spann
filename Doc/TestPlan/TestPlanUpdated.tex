\documentclass[12pt, titlepage]{article}

\usepackage{booktabs}
\usepackage{tabularx}
\usepackage{hyperref}
\hypersetup{
    colorlinks,
    citecolor=black,
    filecolor=blue,
    linkcolor=red,
    urlcolor=blue
}
\usepackage[round]{natbib}

\title{SE 3XA3: Test Plan\\Spann}

\author{Team 8
		\\ Christopher Stokes | stokescd
		\\ Varun Hooda | hoodav
}

\date{\today}

%% Comments

\usepackage{color}

\newif\ifcomments\commentstrue

\ifcomments
\newcommand{\authornote}[3]{\textcolor{#1}{[#3 ---#2]}}
\newcommand{\todo}[1]{\textcolor{red}{[TODO: #1]}}
\else
\newcommand{\authornote}[3]{}
\newcommand{\todo}[1]{}
\fi

\newcommand{\wss}[1]{\authornote{blue}{SS}{#1}}
\newcommand{\ds}[1]{\authornote{red}{DS}{#1}}
\newcommand{\mj}[1]{\authornote{red}{MSN}{#1}}
\newcommand{\cm}[1]{\authornote{red}{CM}{#1}}
\newcommand{\mh}[1]{\authornote{red}{MH}{#1}}

% team members should be added for each team, like the following
% all comments left by the TAs or the instructor should be addressed
% by a corresponding comment from the Team

\newcommand{\tm}[1]{\authornote{magenta}{Team}{#1}}


\begin{document}

\maketitle

\pagenumbering{roman}
\tableofcontents
\listoftables
\listoffigures

\begin{table}[bp]
\caption{\bf Revision History}
\begin{tabularx}{\textwidth}{p{3cm}p{2cm}X}
\toprule {\bf Date} & {\bf Version} & {\bf Notes}\\
\midrule
    Oct. 26, 2016 & Christopher, Varun & Initial test plan\\
    Oct. 31, 2016 & Christopher, Varun & Moved to new template\\
\bottomrule
\end{tabularx}
\end{table}

\newpage

\pagenumbering{arabic}

This document ...

\section{General Information}

\subsection{Purpose}
The purpose of this document is to outline the testing methodologies that will
be used to test the Spann Web IDE application to ensure the application
functions are specified in the software requirements specification document and
to reveal possible bugs in the application.

\subsection{Scope}
The scope of the testing will be the front end JavaScript UI code and the back
end C\# code and SQL queries, as well as the API responses from the server.

\subsection{Acronyms, Abbreviations, and Symbols}
	
\begin{table}[hbp]
\caption{\textbf{Table of Abbreviations}} \label{Table}

\begin{tabularx}{\textwidth}{p{3cm}X}
\toprule
\textbf{Abbreviation} & \textbf{Definition} \\
\midrule
IDE & Application Development Environment\\
API & Application Programming Interface\\
UI & User Interface\\
\bottomrule
\end{tabularx}

\end{table}

\begin{table}[!htbp]
\caption{\textbf{Table of Definitions}} \label{Table}

\begin{tabularx}{\textwidth}{p{3cm}X}
\toprule
\textbf{Term} & \textbf{Definition}\\
\midrule
Term1 & Definition1\\
Term2 & Definition2\\
\bottomrule
\end{tabularx}

\end{table}	

\subsection{Overview of Document}
This document will outline the testing methodologies and various test plans
that the development team will incorporate and use to test the application.

\section{Plan}
	
\subsection{Software Description}
The software, for which the test plan is being written, is an online, web-based,
IDE. The final application will be similar to a
desktop IDE that many developers are familiar with. The application will have
two parts, a front end that will be written in JavaScript and LESS (transpiled
into CSS), and a back end server written in C\# (which uses a SQL server for
persistent data storage).

\subsection{Test Team}
The test team will be made up of Christopher Stokes and Varun Hooda.

\subsection{Automated Testing Approach}

\subsection{Testing Tools}
For the server NUint will be used for unit testing the C\# code.
For the front end we will be using jasmine (a NodeJS module) for unit testing
the JavaScript UI components.
For the server API we will be using postman (a chrome web application).

\subsection{Testing Schedule}
		
\href{../DevelopmentPlan/schedule.png}{See Gantt Chart}

\section{System Test Description}
	
\subsection{Tests for Functional Requirements}

\subsubsection{Front End JavaScript UI Components}
		
\paragraph{JavaScript UI Component Tests}

\begin{enumerate}

\item{Front end JavaScript UI\\}

Type: Dynamic
					
Description: The test will ensure certain properties of each UI components
hold.  Properties such as whether the component is not null, it's fields have
the required values and whether it has required functions and methods.

How test will be performed: Each UI component will have a Jasmine test file.
The Jasmine NodeJS plugin will then be used to execute the tests.
					
\end{enumerate}


\paragraph{Server API Tests}

\begin{enumerate}

\item{Login/Authentication\\}

Type: Dynamic
					
Description: This test will ensure the application is handling user login,
password, username, and authentication in general, correctly.

How test will be performed: This test will be carried out using the Postman
application and a test file written for the Postman application.

\item{Database responses\\}

Type: Dynamic
					
Description: This test will be used to test whether the server correctly
replies with the correct database data. 

How test will be performed: This test will be carried out using the Postman
application and a test file written for the Postman application.

\item{Python Execution\\}

Type: Functional
					
Description: This test will be used to test whether the server replies with a 
correct response to a python program execution. The response should be the output
of the python interpreter exactly to a given input.

How test will be performed: This test will be carried out using the Postman
application and a test file written for the Postman application.

\item{File Management\\}

Type: Functional
					
Description: The application is an IDE, so users will be able to create source
code file. This test will be used to test whether the server is able to handle
files correctly, create them, modify them, move them, delete them.

How test will be performed: This test will be carried out using the Postman
application and a test file written for the Postman application.

\item{User Account\\}

Type: Functional
					
Description: The application will also allow users to set preferences and other
account related details. This test will ensure the back end server handles user
account data correctly.

How test will be performed: This test will be carried out using the Postman
application and a test file written for the Postman application.
					
\end{enumerate}


\subsubsection{Back End C\# Server}

\begin{enumerate}

\item{SQL Queries\\}

Type: Functional
					
Description: 

How test will be performed: 

\item{Front end/back end data transmission\\}

Type: Functional
					
Description: 

How test will be performed: 

\end{enumerate}

\subsection{Tests for Nonfunctional Requirements}

\subsubsection{Front End JavaScript UI}
		
\paragraph{Title for Test}

\begin{enumerate}

\item{test-id1\\}

Type: 
					
Initial State: 
					
Input/Condition: 
					
Output/Result: 
					
How test will be performed: 
					
\item{test-id2\\}

Type: Functional, Dynamic, Manual, Static etc.
					
Initial State: 
					
Input: 
					
Output: 
					
How test will be performed: 

\end{enumerate}

\subsubsection{Back End C\# Server}

...


\section{Tests for Proof of Concept}

\subsection{Front end JavaScript UI}
		
\paragraph{Title for Test}

\begin{enumerate}

\item{test-id1\\}

Type: Functional, Dynamic, Manual, Static etc.
					
Initial State: 
					
Input: 
					
Output: 
					
How test will be performed: 
					
\item{test-id2\\}

Type: Functional, Dynamic, Manual, Static etc.
					
Initial State: 
					
Input: 
					
Output: 
					
How test will be performed: 

\end{enumerate}

\subsection{Server API}

\subsection{Back end C\# server}

	
\section{Comparison to Existing Implementation}	
compare functionality
compare tests

				
\section{Unit Testing Plan}
		
\subsection{Unit testing of internal functions}
		
\subsection{Unit testing of output files}		


\bibliographystyle{plainnat}

\bibliography{SRS}

\newpage

\section{Appendix}

\subsection{Symbolic Parameters}

The definition of the test cases will call for SYMBOLIC\_CONSTANTS.
Their values are defined in this section for easy maintenance.

\subsection{Usability Survey Questions?}

This is a section that would be appropriate for some teams.

\end{document}
