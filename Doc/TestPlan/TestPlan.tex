\documentclass[12pt, titlepage]{article}

\usepackage{booktabs}
\usepackage{tabularx}
\usepackage{hyperref}
\hypersetup{
    colorlinks,
    citecolor=black,
    filecolor=blue,
    linkcolor=red,
    urlcolor=blue
}
\usepackage[round]{natbib}

\title{SE 3XA3: Test Plan\\Spann}

\author{Team 8
		\\ Christopher Stokes | stokescd
		\\ Varun Hooda | hoodav
}

\date{\today}

%% Comments

\usepackage{color}

\newif\ifcomments\commentstrue

\ifcomments
\newcommand{\authornote}[3]{\textcolor{#1}{[#3 ---#2]}}
\newcommand{\todo}[1]{\textcolor{red}{[TODO: #1]}}
\else
\newcommand{\authornote}[3]{}
\newcommand{\todo}[1]{}
\fi

\newcommand{\wss}[1]{\authornote{blue}{SS}{#1}}
\newcommand{\ds}[1]{\authornote{red}{DS}{#1}}
\newcommand{\mj}[1]{\authornote{red}{MSN}{#1}}
\newcommand{\cm}[1]{\authornote{red}{CM}{#1}}
\newcommand{\mh}[1]{\authornote{red}{MH}{#1}}

% team members should be added for each team, like the following
% all comments left by the TAs or the instructor should be addressed
% by a corresponding comment from the Team

\newcommand{\tm}[1]{\authornote{magenta}{Team}{#1}}


\begin{document}

\maketitle

\pagenumbering{roman}
\tableofcontents
\listoftables

\begin{table}[bp]
\caption{\bf Revision History}
\begin{tabularx}{\textwidth}{p{3cm}p{2cm}X}
\toprule {\bf Date} & {\bf Version} & {\bf Notes}\\
\midrule
    Oct. 26, 2016 & Christopher, Varun & Initial test plan\\
    Oct. 30, 2016 & Varun & Moved to new template\\
    Oct. 31, 2016 & Christopher, Varun & Submission for TP-Rev0\\
\bottomrule
\end{tabularx}
\end{table}

\newpage

\pagenumbering{arabic}

\section{General Information}

\subsection{Purpose}
The purpose of this document is to outline the testing methodologies that will
be used to test the Spann Web IDE application to ensure the application
functions are specified in the software requirements specification document and
to reveal possible bugs in the application.

\subsection{Scope}
The scope of the testing will be the front end JavaScript UI code and the back
end C\# code and SQL queries, as well as the API responses from the server.

\subsection{Acronyms, Abbreviations, and Symbols}
	
\begin{table}[hbp]
\caption{\textbf{Table of Abbreviations}} \label{Table}

\begin{tabularx}{\textwidth}{p{3cm}X}
\toprule
\textbf{Abbreviation} & \textbf{Definition} \\
\midrule
IDE & Application Development Environment\\
API & Application Programming Interface\\
UI & User Interface\\
\bottomrule
\end{tabularx}

\end{table}

\subsection{Overview of Document}
This document will outline the testing methodologies and various test plans
that the development team will incorporate and use to test the application.

\section{Plan}
	
\subsection{Software Description}
The software, for which the test plan is being written, is an online, web-based,
IDE. The final application will be similar to a
desktop IDE that many developers are familiar with. The application will have
two parts, a front end that will be written in JavaScript and LESS (transpiled
into CSS), and a back end server written in C\# (which uses a SQL server for
persistent data storage).

\subsection{Test Team}
The test team will be made up of Christopher Stokes and Varun Hooda.

\subsection{Automated Testing Approach}

\subsection{Testing Tools}
For the server NUint will be used for unit testing the C\# code.
For the front end we will be using jasmine (a NodeJS module) for unit testing
the JavaScript UI components.
For the server API we will be using postman (a chrome web application).

\subsection{Testing Schedule}
		
\href{../ProjectSchedule/schedule.png}{See Gantt Chart}

\section{System Test Description}
	
\subsection{Tests for Functional Requirements}

\subsubsection{Front End JavaScript UI Components}
		
\paragraph{JavaScript UI Component Tests}

\begin{enumerate}

\item{Front end JavaScript UI\\}

Type: Dynamic
					
Description: The test will ensure certain properties of each UI components
hold.  Properties such as whether the component is not null, it's fields have
the required values and whether it has required functions and methods.

How test will be performed: Each UI component will have a Jasmine test file.
The Jasmine NodeJS plugin will then be used to execute the tests.
					
\end{enumerate}


\paragraph{Server API Tests}

\begin{enumerate}

\item{Invalid Requests\\}

Type: Functional
					
Description: This test will ensure the system is robust and does not break
if an invalid request is made. An example of an invalid request would be a
non-existent user, non-existent file or project. Since the web application
does not use multiple html pages, there is no need to test invalid urls.

How test will be performed: This test will be carried out using the Postman
application and a test file written for the Postman application.

\item{Login/Authentication\\}

Type: Dynamic
					
Description: This test will ensure the application is handling user login,
password, username, and authentication in general, correctly.

How test will be performed: This test will be carried out using the Postman
application and a test file written for the Postman application.

\item{Database responses\\}

Type: Dynamic
					
Description: This test will be used to test whether the server correctly
replies with the correct database data. 

How test will be performed: This test will be carried out using the Postman
application and a test file written for the Postman application.

\item{Invalid or Malicious Python Execution\\}

Type: Dynamic
					
Description: This test will verify that the system does not have any major
flaws or exploits that the python code could exploit or cause the system to
break. An example would be attempting to write a file that the application
should not be able to access. This test will hopefully result in a failure,
ensuring a user is not able to exploit the system in such a manner.

How test will be performed: This test will be carried out using the Postman
application and a test file written for the Postman application. This test will
also be carried out on the server side to ensure the system is able to handle
the invalid or malicious action successfully.

\item{Python Execution\\}

Type: Functional
					
Description: This test will be used to test whether the server replies with a 
correct response to a python program execution. The response should be the output
of the python interpreter exactly to a given input.

How test will be performed: This test will be carried out using the Postman
application and a test file written for the Postman application.

\item{File Management\\}

Type: Functional
					
Description: The application is an IDE, so users will be able to create source
code file. This test will be used to test whether the server is able to handle
files correctly, create them, modify them, move them, delete them.

How test will be performed: This test will be carried out using the Postman
application and a test file written for the Postman application.

\item{User Account\\}

Type: Functional
					
Description: The application will also allow users to set preferences and other
account related details. This test will ensure the back end server handles user
account data correctly.

How test will be performed: This test will be carried out using the Postman
application and a test file written for the Postman application.
					
\end{enumerate}


\subsubsection{Back End C\# Server}

\begin{enumerate}

\item{SQL Queries\\}

Type: Functional
					
Description: This test will ensure the server generates the correct database
queries once it has received a request from the front end via a web socket.

How test will be performed: This test will be performed using NUint and a test
case written in C\#.

\end{enumerate}

\subsection{Tests for Nonfunctional Requirements}

\subsubsection{Front End JavaScript UI}
		
\paragraph{Look and Feel}

\begin{enumerate}

\item{UI visual inspection\\}

Type: Manual

Description: The purpose of this test will be to manually to inspect the application to
discover visual bug, artifacts, and any other UI imperfections.

How test will be performed: This type of testing is only possible manually,
since it is simply not possible to automatically inspect the UI using a
computer.
					
\item{UI performance\\}

Type: Manual

Description: The purpose of this test is to ensure the application's UI
is fast and responsive. This will help identify any UI performance issues.

How test will be performed: This will also be done manually since it would
be very difficult to programmatically test.

\end{enumerate}

\subsubsection{Server API}

\begin{enumerate}

\item{Account Security\\}

Type: Functional

Description: The purpose of this test is to ensure the only person able to
access a user's account is the actual user only.

How test will be performed: This test will be performed using the Postman
application and test cases written for the Postman application.
					
\item{Response Time\\}

Type: Functional

Description: The purpose of this test will be to ensure the application server
always responds to the user's requests within a justifiable time (depending on
the specific request).  For example, a login should not take more than a single
second to process. The actual python program that will be executed will be
excluded from this since it is not possible to determine the execution time
from statically analysing the python code.

How test will be performed: This test will be performed using the Postman
application and test cases written for the Postman application.

\end{enumerate}

\section{Tests for Proof of Concept}

\subsection{Front end JavaScript UI}
		
\begin{enumerate}

\item{UI components\\}

Type: Dynamic
					
Description: This test will be used to ensure the UI components used in the
proof of concept are functional and have the required properties, functions,
and methods associated with them.
					
How test will be performed: This testing will be done using the Jasmine module
and test files for Jasmine.
					
\item{UI performance\\}

Type: Manual
					
Description: This test will help the developers see if the current design of
the proof of concept provides the performance required in the final
application.
					
How test will be performed: This testing will be manually since it is difficult
to automate it.

\end{enumerate}

\subsection{Server API}

\begin{enumerate}

\item{User account security\\}

Type: Functional
					
Description: This test will be used to check if the current implementation of
the user authentication is sufficiently secure for the final implementation.
					
How test will be performed: This test will be performed using the Postman web
application, as well as, using common manual exploitation techniques such as
session hijacking.
					
\item{Python code execution\\}

Type: Functional
					
Description: The purpose of this test is to ensure the proof of concept, as it
currently functions, returns the correct response (specifically, the exact
response the Python interpreter outputs).
					
How test will be performed: This test will be performed using the Postman web
application and test cases written for it.

\end{enumerate}

\subsection{Back end C\# server}

\begin{enumerate}

\item{Python Execution\\}

Type: Functional
					
Description: The purpose of this test is to ensure the server is able to
receive python code and execute the code using IronPython on the server system.
					
How test will be performed: This test will be performed using NUint and C\#
test cases.
					
\item{SQL Code Generation and Execution\\}

Type: Functional
					
Description: The purpose of this test will be to test if the server is able
to generate the required SQL and able to successfully execute the SQL to save
and retrieve data from the database server.
					
How test will be performed: This test will be performed using NUint and C\#
test cases.

\end{enumerate}
	
\section{Comparison to Existing Implementation}	
The original project, repl.it, on which this project is based supports multiple
languages and a mature and well maintained project. Our application, on the
other hand, focuses on the python programming language specifically. So the
scope of our project is much narrower than the original.

A narrower scope means we are also able to test our application more thoroughly.
The original project seems to focus more on test higher level features,
specifically, whether the front and back end have a connection, data is being
sent back and forth. Our project will go more into the specific features and
perform more extensive tests to verify the functionality and the non-functional
aspects of the application.

				
\section{Unit Testing Plan}
		
\subsection{Unit testing of internal functions}

\subsection{C\#}
In order to test the internal functions and logic of the C\# server, the
reflective capabilities of C\# will be utilised. NUnit utilises reflection to
provide access to internal classes not normally accessible, this is necessary
as by default C\# classes are private to all but the namespace. To extend this
capability reflection will be used to access the private member properties, and
functions allowing direct calls to these elements and direct access to their
return values. This will greatly reduce the amount of mocking required to
execute the tests, and overall providing simpler tests and more rugged and
reliable test which are not easily broken. 

\subsection{JavaScript} In order to test the JavaScript modules, multiple tests
patterns will need to be used due to the nature of the code. 

The first type applies to all the custom UI components of the Spann UI engine,
where the private members are located directly in the object returned from the
module. The internal parts of the components are located in an object named
\_private on the module objects. This is a consequence of patterns used to allow
inheritance in JavaScript and lack of a dependency manager for the UI
components, increasing performance.  

In comparison, the remaining code in the Spann client uses the revealing module
pattern as well as AMD with RequireJS. This combination means a simple
management of dependencies but as a consequence, there is no way to access the
components not revealed from the module. Therefore, all modules will reveal a
method named \_getInternals which will return all the internal methods allowing
them to be tested. 

		
\subsection{Unit testing of output files}		
N/A


\bibliographystyle{plainnat}

\bibliography{SRS}

\end{document}
