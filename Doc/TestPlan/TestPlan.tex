\documentclass{article}

\usepackage{booktabs}
\usepackage{tabularx}
\usepackage{graphicx}

\title{SE 3XA3: Test Plan\\Spann}

\author{Team 8
		\\ Christopher Stokes | stokescd
		\\ Varun Hooda | hoodav
}

\date{}

%% Comments

\usepackage{color}

\newif\ifcomments\commentstrue

\ifcomments
\newcommand{\authornote}[3]{\textcolor{#1}{[#3 ---#2]}}
\newcommand{\todo}[1]{\textcolor{red}{[TODO: #1]}}
\else
\newcommand{\authornote}[3]{}
\newcommand{\todo}[1]{}
\fi

\newcommand{\wss}[1]{\authornote{blue}{SS}{#1}}
\newcommand{\ds}[1]{\authornote{red}{DS}{#1}}
\newcommand{\mj}[1]{\authornote{red}{MSN}{#1}}
\newcommand{\cm}[1]{\authornote{red}{CM}{#1}}
\newcommand{\mh}[1]{\authornote{red}{MH}{#1}}

% team members should be added for each team, like the following
% all comments left by the TAs or the instructor should be addressed
% by a corresponding comment from the Team

\newcommand{\tm}[1]{\authornote{magenta}{Team}{#1}}


\begin{document}

\begin{table}[hp]
\caption{Revision History} \label{TblRevisionHistory}
\begin{tabularx}{\textwidth}{llX}
\toprule
\textbf{Date} & \textbf{Developer(s)} & \textbf{Change}\\
\midrule
    Oct. 26, 2016 & Christopher, Varun & Initial test plan\\
\bottomrule
\end{tabularx}
\end{table}

\newpage

\maketitle

\section{Structural Tests}
In order to ensure our application functions as expected we will perform
structural or white-box testing. Since our application has two major parts,
the front end UI and the back end server, we will need different ways to test
each.

To test the front end, structurally, we will make use to XXXXX. This will allow
us to be confident our front end JavaScript code is functioning as we indent it
to and identify any flaws or bugs ??therein??.

To test the back end, functionally, we will make use of XXXX. XXXX will allow
us to be confident our back end C\# code and SQL queries are functioning as
we indent and identify any flaws or bugs ??therein??.

\section{Functional Tests}
Testing the front end and back end code using a white-box testing approach
allows us to validate our code, but in order to test the functionality as
specified by the Software Requirements Document we will need to use a
functional testing approach.

To test the front end, functionally, we will make use to XXXXX.
XXXX will allow us to test the functionality, as specified by the software
requirements document without knowing the exact code that is performing
underneath.

To test the back end, functionally, we will make use of XXXX. XXXX will allow
us to perform API requests and validate the responses we get back, again,
without concern for the exact code that is computing the results.

\section{Unit Tests}
Unit testing will be performed for both the front end and the back.
Specifically, for the front end we will be using XXXX for the individual UI
components. And for the back end we will be using XXXX which allows us to
perform unit tests on individual blocks of server code.

\section{Static Tests}

\section{Dynamic Tests}

\section{Manual Tests}
Manual testing will ??begin to be performed once the application is nearing its
final form?? and the major features and components are in place. To perform
manual testing, the application will attempt to be used in its intended form
and the purpose of the tests will be to ensure the application is both usable
and perform as expected during regular usage.

\section{Automated Tests}

\section{Performing Tests}
how tests will be performed

\section{Performing Automated Tests}
how automated tests will be performed

\section{System Specific Tests}

\section{Usability Tests}

\section{Testing Schedule}

\section{Code Coverage}

\end{document}
