\documentclass[12pt, titlepage]{article}

\usepackage{booktabs}
\usepackage{tabularx}
\usepackage{hyperref}
\hypersetup{
    colorlinks,
    citecolor=black,
    filecolor=black,
    linkcolor=red,
    urlcolor=blue
}
\usepackage[round]{natbib}

\title{SE 3XA3: Test Plan\\Spann}

\author{Team 5
  \\ Christopher Stokes | stokescd
  \\ Varun Hooda | hoodav
}

\date{\today}

%% Comments

\usepackage{color}

\newif\ifcomments\commentstrue

\ifcomments
\newcommand{\authornote}[3]{\textcolor{#1}{[#3 ---#2]}}
\newcommand{\todo}[1]{\textcolor{red}{[TODO: #1]}}
\else
\newcommand{\authornote}[3]{}
\newcommand{\todo}[1]{}
\fi

\newcommand{\wss}[1]{\authornote{blue}{SS}{#1}}
\newcommand{\ds}[1]{\authornote{red}{DS}{#1}}
\newcommand{\mj}[1]{\authornote{red}{MSN}{#1}}
\newcommand{\cm}[1]{\authornote{red}{CM}{#1}}
\newcommand{\mh}[1]{\authornote{red}{MH}{#1}}

% team members should be added for each team, like the following
% all comments left by the TAs or the instructor should be addressed
% by a corresponding comment from the Team

\newcommand{\tm}[1]{\authornote{magenta}{Team}{#1}}


\begin{document}

\maketitle

\pagenumbering{roman}
\tableofcontents
\listoftables
%\listoffigures

\begin{table}[bp]
\caption{\bf Revision History}
\begin{tabularx}{\textwidth}{p{3cm}p{2cm}X}
\toprule {\bf Date} & {\bf Version} & {\bf Notes}\\
\midrule
  Dec 7, 2016 & 1.0 & Added content\\
\bottomrule
\end{tabularx}
\end{table}

\clearpage

\pagenumbering{arabic}

This document summaries the various testing strategies employed in this project
to ensure the application meets the specified requirements, both functional
and non-functional. Specifically, this document will outline all the tests that
were performed, along with the results of each test, and any changes to the
application that came from those results.

\section{Functional Requirements Evaluation}

  \subsubsection{Front End Tests}
    \begin{enumerate}

      \item{SHA512\\} \label{sha}

      Initial State: Function, has no state.

      Input: UI data objects with a string to be hashed.

			Expected: Correct hash string,

			Result: PASS -- The module results the correct hash string.

      \item{Encryption utils\\} \label{utils}

      Initial State: Function, has no state.

      Input: UI data objects of a string and a salt.

      Expect: UI response objects with the hash string

      Result: PASS -- Module returns correctly formatted object with the hash string.

      \item{Login Screen\\} \label{login}

      Initial State: Function, has no state.

      Input: Set of correct user credentials and a set of incorrect user credentials.

			Expect: (returns true) Successful authentication with correct credentials
				and (return false) no authentication with incorrect credentials.
                
			Result: PASS -- module does not allow unauthorized access when given
				incorrect username and password, but does allow access when correct username
				and password are supplied.
                
    \end{enumerate}

  \subsubsection{Server API Tests}

    \begin{enumerate}

      \item{Invalid Requests\\} \label{invalid}

      Type: Functional, Dynamic

      Initial State: Function, has no state.

      Input: Invalid user authentication request, invalid file/project request.

      Expect: Response object from server indicating that the request is invalid.
                
			Result: PASS -- Server returns object indicating an incorrect/invalid
				request is made and no action can be performed.

    \end{enumerate}

  \subsubsection{Back End C\# Server}

    \begin{enumerate}

      \item{SQL Queries\\} \label{sql}

      Initial State: Function, has no state.

      Input: Database request object.

      Expect: A valid SQL query that corresponds to the request.
                
			Result: PASS -- The module provides a correct and valid SQL query.

			\paragraph{Example SQL Tests}
			\begin{enumerate}
				\item Create\\

				{\bf Test 1\\}

				Name: Domain Model Object Data\\

				Output: valid SQL code for creating that object in the database.\\

				Expected Results: SQL code with random data inserted into correct spot\\

				Result: PASS\\

				\item Update\\

				{\bf Test 1\\}

				Name: Test Generate Update With Where\\

				Initial State: No initial state. Code is a Pure Function and has no internal sate data.\\

				Input:\\
				1. DomianModel object data.\\
				2. Inline LINQ code for the C\# representation of the where statement\\

				Output: valid SQL code for updating that object in the database with
				the C\# LINQ code transformed into an expression tree then into the correct
				select statement.\\

				Result: PASS\\

				{\bf Test 2:\\}

				Name: Test Generate Create With ID\\

				Initial State: No initial state. Code is a Pure Function and has no internal sate data.\\

				Input:\\
				1. DomianModel object data.\\
				2. ID of object to update\\

				Output: valid SQL code for removing that object in the database with
				select clause for that ID.\\

				Expected Results: SQL code with random data inserted into correct spot.\\

				Result: PASS\\

				\item Load\\

				{\bf Test 1\\}

				Name: Test Generate Load With Where\\

				Initial State: No initial state. Code is a Pure Function and has no internal state data\\

				Input:\\
				1. Domain Model Object Data.\\
				2. Inline LINQ code for the C\# representation of the where statement.\\

				Output: valid SQL code for loading that object in the database
				with the C\# LINQ code transformed into an expression tree then into the
				correct select statement.\\

				Result: PASS\\

				{\bf Test 2:\\}

				Name: Test Generate Load With ID\\

				Initial State: No initial state. Code is a Pure Function and has no internal sate data.\\

				Input:\\
				1. Domain Model object data.\\
				2. ID of object to update\\

				Output: valid SQL code for removing that object in the database with
				select clause for that ID.\\

				Expected Results: SQL code with random data inserted into correct spot.\\

				Result: PASS

				\item Delete\\

				Name: Test Generate Delete With Where\\

				Initial State: No initial state. Code is a Pure Function and has no internal sate data.\\

				Input:\\
				1. Domain Model object data.\\
				2. Inline LINQ code for the C\# representation of the where statement\\

				Output: valid SQL code for removing that object in the database with
				the C\# LINQ code transformed into an expression tree then into the correct
				select statement.\\

				Expected Results: SQL code with random data inserted into correct spot.\\

				Result: PASS\\

			\end{enumerate}


      \item{Domain Model Tests\\} \label{dto}

      Initial State: Function, has no state.

      Input: Domain Model Transfer Object.

			Expect: Data Model (DM) object with the correct corresponding metadata.

			Result: PASS -- Module returns a correct DM with the correct metadata.

    \end{enumerate}

  \subsubsection{Back end C\# server}

    \begin{enumerate}

    \item{Python Execution\\} \label{exec}

    Initial State: Function, has no state.

    Input: String of valid python code and a string of invalid python code.

    Expect: String containing the expected output for a valid piece of python code
			and a string containing the expected error's for a invalid piece of python code.
              
		Result: PASS -- Module returns expected code output for valid code and a
		string of error/exceptions/stack trace for an invalid string of code.
              
    \end{enumerate}


\section{Nonfunctional Requirements Evaluation}

  \subsubsection{Front End JavaScript UI}

    \begin{enumerate}

    \item{UI performance\\} \label{ui}

    Initial State: Application running inside browser.

    Input: User input using mouse clicks and keyboard input.

    Expect: The application should feel responsive to any user actions and
		 actions performed should be performed so that the user is kept engaged.
              
		Result: PASS -- The application's response times are unnoticeable while in
		use. The longest delay in any action performed is when the actual python
		code is sent to the server to be executed (the delay is a result of the actual
		python code execution, as opposed to any request processing delay).

		\end{enumerate}
              
    \paragraph{Look and Feel}

    \begin{enumerate}

    \item{UI visual inspection\\} \label{visual}

    Initial State: Application running inside browser.

    Input: Mouse inputs for navigating the application and keyboard for any textual
    input for the purposes of seeing text/code.

		Expect: The rendered DOM (html elements) should be as the developer intended. There should not be
		any anomalies, bugs or incorrect properties in any of the elements rendered to
		the screen. The results should be as similar as possible on different browsers
		(browsers inherently render elements differently, so results should be as
		minimal as possible across browsers).

		Result: PASS -- All elements are rendered as the developer who designed them
		intended. The rendering across browsers has no noticeable differences.

    \end{enumerate}

  \subsubsection{API /Back End Tests}
  \paragraph{Stress Test}
      
    \begin{enumerate}

    \item{API/Server Stress\\} \label{stressTest}

    Initial State: Function, has no state.

    Input: String of python code from multiple concurrent connections.

		Expect: The server should process each request and respond with the
		correct result with minimal delay. Increasing the stress on the
		application should have a proportional delay in response time.

		Result: PASS -- The application handles a large number of concurrent requests
		without noticeable delays in performance/response time. It was noted that
		the results do differ when a less powerful machine is used to perform the tests
		and host the server, since the hardware is less powerful and can't handle the
		stresses of newer/more powerful machines. Thus, Modern machines are well
		capable of hosting the application without any performance issues.

    \end{enumerate}

  \subsubsection{Usability}
  The usability of the project was manually determined using the usability
  survey questions (as specified in the Test Plan document).

  After manually using the application as a regular user would, it was deemed
  that the application is quite usable and there are not any aspected that hinder
  the application or present a scenario/situation in which the application fails
  to be usable (under regular, expected usage).

\section{Comparison to Existing Implementation}	

The existing implementation did not have a set of tests to compare the results
against. But the manual comparison of the two implementation can be made.

\paragraph{Functionality\\}

Both projects has very similar functionality. The major difference is that
Spann supports projects and multi file projects while the existing
implementation does not. The python execution and results of the same code on
both platform are, as expected, identical.

\paragraph{Performance\\}

Both implementations a fast and responsive. It can be noted, since the existing
implementation is running on what can be assumed to be faster hardware, the
python code executes faster that on Spann.

\paragraph{Stress Test\\}

The existing implementation is hosted in a public facing server and it can be
assumed thousands of concurrent requests can be made without any delays. Testing this
would be very difficult for the developers of this, Spann, project.

As noted in the stress test section \ref{stressTest}, this application is able
to support concurrent requests without noticeable delays or issues regarding
performance. But the performance overall is dictated by the hardware the
application is running on, hence, running on a modern laptop computer, this
application can't support thousands of concurrent requests. But, should this
application be hosted on a more powerful server, a larger number of concurrent
requests could be supported. Testing on a powerful server is currently not
feasible for this project and team.

\section{Unit Testing}
There were two sets of unit tests performed. The Jasmine, JavaScript based,
unit tests for front-end facing code were performed using the NodeJS Jasmine
module. The Nunit, C\# based, unit test for server components and modules were
performed using C\# and the Nunit testing framework.

The Jasmine tests using the "spec" prefix to specify which component/module
they are testing. A configuration file is used to indicate the location of the
project code and the test code. Jasmine uses this configuration file to
automatically perform the tests.

The C\# tests are performed using Visual Studio. This method of tests are very
similar to the well known Java JUnit method of testing. Visual Studio handles
the majority of the work in regards to testing, the developer needs only write
the test and specify the code to be run before and after each test/suite of
tests (as it is in JUnit).

\section{Changes Due to Testing}

Summary of the changes as a result of testing,

\begin{itemize}
\item Refactoring UI JavaScript UI components to allow code inside callbacks
and inline function definitions, to be exposed to the testing framework.
\item Refactoring server C\# code for better modularity and individual testing.
Reducing the coupling between pieces of code to allow tests to test individual
components.
\item Editing CSS/UI specification in JavaScript to allow for elements to be
rendered as similarly as the browsers allow when the application is running on
different browsers.
\item Handling user input in the python console interpreter. The python console
interpreter is a heavily modified version of the regular editor used throughout
this application for writing code. The testing, manual testing since that
is the only feasible means of testing this, revealed subtle bug in the way
the components performed and handled input.
\item UI/overall front end design to accommodate when the framework allowed
and the different browsers allowed to be rendered correctly.
\end{itemize}

\section{Automated Testing}
The automated testing was performed using, as noted previously, Jasmine and
NUnit. The other automated testing that was performed, API testing, was
performed using the Postman chrome web application. Postman allows the developers
to write scripts that perform automated tests for the API.

All the front end tests for UI components were performed using Jasmine. All the
API tests were performed using Postman. All the server tests for C\# server
components was performed using NUnit.
		
\section{Trace to Requirements} \label{SecTM}

\begin{table}
\centering
\begin{tabular}{p{0.3\textwidth} p{0.4\textwidth}}
\toprule
\textbf{Req.} & \textbf{Modules}\\
\midrule
    Mode & Login Screen \ref{login}\\
				 & SHA512 \ref{sha}\\
				 & Encryption Utils \ref{utils}\\
    Editing & UI Performance \ref{ui}\\
				 &  UI Visual Inspection \ref{visual}\\
    Editing Support & UI Performance \ref{ui}\\
			   &  UI Visual Inspection \ref{visual}\\
    File Handling & Invalid Request \ref{invalid}\\
				 & SQL Queries \ref{sql}\\
				 & Domain Model \ref{dto}\\
    Code Execution & Python Execution \ref{exec}\\
				 & Domain Model \ref{dto}\\
				 & API/Server Stress \ref{stressTest}\\
    Shell Interpreter & Python Execution \ref{exec}\\
				 & API/Server Stress \ref{stressTest}\\
				 & UI Performance \ref{ui}\\
				 & UI Visual Inspection \ref{visual}\\
    Networking & Invalid Request \ref{invalid}\\
				 & Stress Test \ref{stressTest}\\
    Accounts & Invalid Request \ref{invalid}\\
				 & Domain Model \ref{dto}\\
    Account Management & Invalid Request \ref{invalid}\\
				 & Domain Model \ref{dto}\\
    Look and Feel & UI Performance \ref{ui}\\
				 & UI Visual Inspection \ref{visual}\\
    Usability & UI Performance \ref{ui}\\
				 & UI Visual Inspection \ref{visual}\\
    Performance & UI Performance \ref{ui}\\
				 & API/Server Stress \ref{stressTest}\\
    Security & Login Screen \ref{login}\\ 
				 & SHA512 \ref{sha}\\
				 & Encryption Utils \ref{utils}\\
    Health and Safety & UI Visual Inspection \ref{visual}\\
\bottomrule
\end{tabular}
\caption{Trace to Requirements }
\label{TblRT}
\end{table}

\clearpage
		
\section{Trace to Modules}		

\begin{table}
\centering
\begin{tabular}{p{0.3\textwidth} p{0.4\textwidth}}
\toprule
\textbf{Req.} & \textbf{Modules}\\
\midrule
	Hardware-Hiding Module & UI Performance \ref{ui}\\
				 & UI Visual Inspection \ref{visual}\\
				 & API/Server Stress \ref{stressTest}\\
	\\API Controllers & SQL Queries \ref{sql}\\
				 & Domain Model \ref{dto}\\
				 & Python Execution \ref{exec}\\
	\\Python Console & API/Server Stress \ref{stressTest}\\
				 & Python Execution \ref{exec}\\
				 & UI Performance \ref{ui}\\
				 & UI Visual Inspection \ref{visual}\\
	\\Python Console Manager & Domain Model \ref{dto}\\
				 & Python Execution \ref{exec}\\
	\\Python Runners & Python Execution \ref{exec}\\
	\\Repository Model & Domain Model \ref{dto}\\
				 & SQL Queries \ref{sql}\\
	\\Domain Models & Domain Model \ref{dto}\\
				 & SQL Queries \ref{sql}\\
	\\UI Screens & UI Performance \ref{ui}\\
				 & UI Visual Inspection \ref{visual}\\
\bottomrule
\end{tabular}
\caption{Trace to Modules }
\label{TblRT}
\end{table}

\newpage

\section{Code Coverage Metrics}
Visual Studio was used to measure code coverage for all C\# code. Visual studio
reports 26\% coverage. A reason for not having a higher coverage is that a lot of
the testable code inherits from a select from classes. A better approach to
testing, in reflection, would be to have tests that incrementally tests each
class and become more specific with each child class. This would result in
better code coverage and a more through test suite.

For the API test, this project achieved 100\% code coverage, as reported by
Postman (the chrome application this project uses for testing the API). API for
this project is very simple and quite small. This allows the developers to test
all aspects of the API and achieved 100\% coverage.

The front end JavaScript was not measured for code coverage due to infeasibility.
The front end relies heavily on the JJS UI framework. Measuring code coverage
is also made more difficult by the nature of JavaScript and the heavy use of
closures in both this project's JavaScript code and in JJS's code. A rough
estimate of code coverage by inspection would yield a number of around 25\%, but
this should not be considered a reliable figure.

\bibliographystyle{plainnat}

\bibliography{SRS}

\end{document}
