\documentclass{article}

\usepackage{booktabs}
\usepackage{tabularx}

\title{SE 3XA3: Problem Statement\\Spann}

\author{Team 8
		\\ Christopher Stokes | stokescd
		\\ Varun Hooda | hoodav
}

\date{}

%% Comments

\usepackage{color}

\newif\ifcomments\commentstrue

\ifcomments
\newcommand{\authornote}[3]{\textcolor{#1}{[#3 ---#2]}}
\newcommand{\todo}[1]{\textcolor{red}{[TODO: #1]}}
\else
\newcommand{\authornote}[3]{}
\newcommand{\todo}[1]{}
\fi

\newcommand{\wss}[1]{\authornote{blue}{SS}{#1}}
\newcommand{\ds}[1]{\authornote{red}{DS}{#1}}
\newcommand{\mj}[1]{\authornote{red}{MSN}{#1}}
\newcommand{\cm}[1]{\authornote{red}{CM}{#1}}
\newcommand{\mh}[1]{\authornote{red}{MH}{#1}}

% team members should be added for each team, like the following
% all comments left by the TAs or the instructor should be addressed
% by a corresponding comment from the Team

\newcommand{\tm}[1]{\authornote{magenta}{Team}{#1}}


\begin{document}

\begin{table}[hp]
\caption{Revision History} \label{TblRevisionHistory}
\begin{tabularx}{\textwidth}{llX}
\toprule
\textbf{Date} & \textbf{Developer(s)} & \textbf{Change}\\
\midrule
    Sept. 26, 2016 & Christopher, Varun & Initial development plan\\
    Sept. 28, 2016 & Varun & Formatting, Introduction and Git Workflow\\
\bottomrule
\end{tabularx}
\end{table}

\newpage

\maketitle

This document outlines some key points regarding the development of this
project. The points discussed related to the way the team will work (meetings,
workflow, communication) and how the project will undergo development and
progress as time goes on.

\section{Team Meeting Plan}
The team will meet on a weekly basis on Tuesday afternoons on campus (exact
location and time is up to the discretion of the team members). The meetings
will allow the team to express any concerns, discuss upcoming
deadlines/milestones, and discuss the work plan for the upcoming week. Team
members will alternate as the chair for each meeting. The char will be
responsible for creating an appropriate agenda for the weeks meeting and for
directing the meeting.

\section{Team Communication Plan}
The primary means of communication will be Google Hangouts. Issue tracking on
gitlab will be used for formally discussing any issues with the project.

\section{Team Member Roles}
There will be no team leader due the small team size. A team member may lead
the team for a particular task if the team member is experienced in that
particular task.
\begin{description}
  \item[Christopher Stokes] Primarily work on backend server code, as well as
    some work on the custom framework that will be used and on some frontend
    code. Will be the expert on the technologies used on this project.
  \item[Varun Hooda] Primarily work on frontend code, as well as on the custom
    framework. Will be the expert on \LaTeX
\end{description}

\section{Git Workflow Plan}
Git and Gitlab will be used to manage the project's documentation and code
base. The team will use a single repository (no forks) with all developers
contributing to the same code base. Git branches will be used to reduce
conflicts between different incomplete features The team will attempt to commit
and push changes frequently. Labels will be used to differential or highlight
particular milestones.

\section{Proof of Concept Demonstration Plan}
During the demo, the majority of problems could arise from issues connecting to the server or database due to firewall or network preferences. To ensure that this does not occur during the demo, the server and database will be hosted locally on the computer running the web app. This minimises all connection issues as the connection is local so the firewall and outside network will have no effect on the connection.  

\section{Technology}
The project will use modern web technologies for the frontend: HTML5, CSS3
(compiled from LESS) and JavaScript. The backend server will run C\# code and
use the PostgreSQL object-relational database management system (ORDBMS) to
store and retrieve data. A custom JavaScript web framework will also be used to
create the frontend website (to avoid manually writing HTML).

\section{Coding Style}
\begin{itemize}
  \item ECMA5 Script Standard for JavaScript
  \item LESS Standard for LESS files
  \item Microsoft Coding Conventions for C\#
\end{itemize}

\section{Project Schedule}

Provide a pointer to your Gantt Chart.

\section{Project Review}

\end{document}
